\documentclass[10pt,xcolor=svgnames]{beamer} %Beamer
\usepackage{palatino} %font type
\usepackage{tikz}
\usetikzlibrary{calc}
%\usepackage[style=verbose,backend=biber]{biblatex}
\usepackage[style=authoryear]{biblatex}
\renewcommand*{\nameyeardelim}{\addcomma\addspace}

\usefonttheme{metropolis} %Type of slides
\usefonttheme[onlymath]{serif} %font type Mathematical expressions
\usetheme[progressbar=frametitle]{metropolis} %This adds a bar at the beginning of each section.

\usepackage{appendixnumberbeamer} %enumerate each slide without counting the appendix
\setbeamercolor{progress bar}{fg=Orange!70!Coral} %These are the colours of the progress bar. Notice that the names used are the svgnames
\setbeamercolor{title separator}{fg=DarkSalmon} %This is the line colour in the title slide
\setbeamercolor{structure}{fg=black} %Colour of the text of structure, numbers, items, blah. Not the big text.
\setbeamercolor{normal text}{fg=black!87} %Colour of normal text
\setbeamercolor{alerted text}{fg=DarkRed!60!Gainsboro} %Color of the alert box
\setbeamercolor{example text}{fg=Maroon!70!Coral} %Colour of the Example block text


\definecolor{backgroundExp}{RGB}{127.5,127.5, 127.5} %These are the colours of the background. Being this the main combination and so one. 
\setbeamercolor{palette primary}{bg = backgroundExp, fg=white}
\setbeamercolor{palette secondary}{bg=NavyBlue!50!DarkOliveGreen, fg=white}
\setbeamercolor{palette tertiary}{bg=NavyBlue!40!Black, fg= white}
\setbeamercolor{section in toc}{fg=NavyBlue!40!Black} %Color of the text in the table of contents (toc)

%These next packages are the useful for Physics in general, you can add the extras here. 
\usepackage{amsmath,amssymb}
\usepackage{slashed}
\usepackage{relsize}
\usepackage{caption}
\usepackage{subcaption}
\usepackage{multicol}
\usepackage{booktabs}
\usepackage[scale=2]{ccicons}
\usepackage{pgfplots}
\usepgfplotslibrary{dateplot}
\usepackage{geometry}
\usepackage{xspace}


\definecolor{mpigreen}{HTML}{007977}
\setbeamercolor{frametitle}{bg=mpigreen}

\definecolor{experimentBackground}{RGB}{127.5,127.5,127.5}

\newcommand{\themename}{\textbf{\textsc{bluetemp}\xspace}}%metropolis}}\xspace}
\titlegraphic{%
  \begin{picture}(0,0)
    \put(315,-200){\makebox(0,0)[rt]{\includegraphics[width=1cm]{logos/logoCPI.jpeg}}}
  \end{picture}
  \begin{picture}(0,0)
    \put(280,-200){\makebox(0,0)[rt]{\includegraphics[width=4cm]{logos/logoMPI.png}}}
  \end{picture}
  \begin{picture}(0,0)
    \put(80,-200){\makebox(0,0)[rt]{\includegraphics[width=3cm]{logos/logoUNI.png}}}
  \end{picture}
  }
  

\title{Characterizing Temporal Dynamics of Visual Perceptual Grouping}
\author[Name]{Mehmet Yörüten \\
            \textbf{Supervised by: } Shuchen Wu, Felix Wichmann, Eric Schulz \\ } %With inst, you can change the institution they belong
\institute[uni]{\textit{M.Sc Student of Neural Information Processing} \\ University of Tübingen}
\date{} 

%References file

\begin{document}

\maketitle
\metroset{titleformat frame=smallcaps} %This changes the titles for small caps

\begin{frame}{Getting Segments - \textit{Normalized Min Cut}}
\begin{figure}
    \hspace*{-1em}
    \vspace*{-1em}
    \begin{tikzpicture}
      [
        grow                    = right,
        sibling distance        = 14em,
        level distance          = 17em,
        edge from parent/.style = {draw, -latex},
        every node/.style       = {font=\footnotesize},
        scale = 0.46,
        sloped
      ]
      \node(0) {\includegraphics[width=0.2\textwidth]{pictures/grid_init4.png}}
            child {node (seg1) {\includegraphics[width=0.2\textwidth]{pictures/seg_2_cut1.png}}
                }
            child {node (seg2) {\includegraphics[width=0.2\textwidth]{pictures/seg_3_cut1.png}}
                child {node (segLastBottom) {\includegraphics[width=0.2\textwidth]{pictures/seg_4_cut2.png}}}
                child {node (segLastTop) {\includegraphics[width=0.2\textwidth]{pictures/seg_5_cut2.png}}
                    child {node (seg7) {\includegraphics[width=0.2\textwidth]{pictures/seg_7_cut3.png}}}
                    child {node (seg6) {\includegraphics[width=0.2\textwidth]{pictures/seg_6_cut3.png}}}
                }
                };
        \node (Cut1) [above of=seg2, yshift=0.7cm] {Cut 1};
        \node (Cut2) [above of=segLastTop, yshift=0.5cm] {Cut 2};
        \node (Cut3) [above of=seg6, yshift=0.5cm] {Cut 3};

        
        \draw[dotted, ultra thick, transform canvas={xshift=+3.8em}] ($(Cut1)$) -- ($(seg1)$) node (expS) {Short Exposure Time}; 
        \draw[dotted, ultra thick, transform canvas={xshift=+3.8em}] ($(Cut3)$)++(0,+30pt) -- ($(seg7)$) node (expL) {Long Exposure Time} ; 
        
        
        %\draw[dashed,rounded corners=2, red]($(Cut1)+(-2.95,+0.95)$)rectangle($(seg1)+(3,-2.3)$); % Draw rectangles from northwest to southeast
        %\draw[dashed,rounded corners=2, blue]($(Cut1)+(-2.95,+0.95)$)rectangle($(seg7)+(3,-2.3)$); % Draw rectangles from northwest to southeast
        %\draw[dotted, ultra thick, transform canvas={xshift=+3.8em}] ($(Cut1)$) ++(0,+150pt) -- ($(seg1)$) node (expS) {Short Exposure Time}; 
        %\draw[dotted, ultra thick, transform canvas={xshift=+3.8em}] ($(Cut3)$)++(0,+30pt) -- ($(seg7)$) node (expL) {Long Exposure Time} ; 
        
    \end{tikzpicture}
    \caption{First three steps of segmentation}
    \label{fig:gridSegmentationExample}
    
\end{figure}
\end{frame}

\end{document}